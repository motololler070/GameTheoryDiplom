\textit{Определение} \textbf{Франшиза} -  форма бизнеса, при которой владелец бренда или права на товары или услуги (франчайзер) предоставляет индивидуальным предпринимателям или компаниям  право на использование своего бренда, бизнес-модели и поддержку в обмен на определенные платежи, обычно в форме единовременного взноса и ежемесячных платежей  за использование бренда и поддержку. \cite{caves2014franchising}

Основные характеристики франшизы включают в себя:
\begin{itemize}
    \item Предоставление права на использование бренда, товарных знаков, товаров и услуг, а также бизнес-модели франчайзера; 
    \item Предоставление поддержки и обучения франчайзи в запуске и ведении бизнеса;
    \item Систематизированная поддержка со стороны франчайзера, такая как маркетинговая поддержка, обновление бизнес-модели и поставка товаров и услуг;
    \item Франчайзер требует от франчайзи соблюдения определенyых стандартов и процедур, чтобы обеспечить единый стандарт качества бренда во всех точках продажи.
\end{itemize}

Франчайзер получает возможность расширения своего бизнеса без необходимости вложения собственных средств и управления новыми точками продажи, в то время как франчайзи получает доступ к узнаваемому бренду, установленной бизнес-модели и поддержке со стороны опытного партнера.
