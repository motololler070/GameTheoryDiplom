Управление строится на построение договоренности между сторонами, которые устанавливаются через количественные и качественные показатели. Опишем методологию.

\textit{Определение} \textbf{KPI (Key Performance Indicator)} - это ключевой показатель эффективности, который используется в бизнесе для измерения достижения организационных целей и оценки производительности на основе конкретных метрик. KPI представляют собой количественные или качественные показатели, которые помогают компаниям оценить, насколько успешно они достигают своих целей и выполнение стратегических задач.

Примеры KPI включают выручку от продаж, чистую прибыль, уровень удовлетворенности клиентов, долю рынка, процент выполнения проектов в срок и т. д. Эти показатели помогают компаниям определить, где они находятся относительно своих целей и принимать соответствующие меры для улучшения производительности и достижения успеха.

\textit{Определение} \textbf{SLA} (Service Level Agreement) - это формальный документ, заключаемый между поставщиком услуг  и клиентом, который устанавливает обязательства и ожидания сторон относительно качества предоставляемых услуг. 

Основные компоненты SLA обычно включают в себя четкое определение предоставляемой услуги, включая область применения, функциональные возможности и ограничения. Определение обязательств и ответственности поставщика и клиента, включая процедуры уведомления, реагирования на инциденты и решения споров. SLA задает уровни сервиса (SL) - это ключевые критерии и метрики, которые определяют качество предоставляемой услуги в рамках.

Применительно к сфере поставок можно выделить SLA следующий набор целевых показателей:
\begin{itemize}
    \item Время выполнения заказа
    \item Качество товара
    \item Своевременность доставки
    \item Уведомления об изменениях
\end{itemize}

Сервис различает по уровню критичности. В порядке возрастания:
\begin{itemize}
    \item Наилучшие пожелания
    \item Стандартный уровень
    \item Критически важный уровень
\end{itemize}

Матрица обязательств (SL) запишется как:


%
\begin{center}
\begin{tabular}{ |c|c|c|c| }
 \text{Время выполнения заказа} & \text{Качество товара} & \text{Своевременность доставки} & \text{Уведомления об изменениях} \\ 
 \hline
 \text{95\% заказов в течение 24 часов} &  \text{менее 1\% брака} & \text{100\% в срок} & немедленно\\ 
 \text{80\%  в течение 48 часов} & \text{менее 5\% брака}  & \text{95\% заказов в срок}  & \text{в течение 24 часов} \\ 
 \text{без гарантий} & \text{без гарантий} & \text{без гарантий} & \text{без гарантий} \\ 
 \hline
\end{tabular}
\end{center}


\textit{Определение} \textbf{IPO}(Initial public offering) 


Предметно раскрыта в статьей \cite{leland1977informational}/

