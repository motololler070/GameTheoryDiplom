Одной из наиболее влиятельных и авторитетных работ является научный труд американских экономистов Пола Дугласа и Чарльза Кобба  
"A theory of production" \cite{cobb1928theory}.

Эта работа представляет собой одну из первых попыток формализовать связь между входами и выходами в производственном процессе и стала отправной точкой для развития теории производства и анализа затрат. В книге авторы представили известную сегодня \textit{функцию Кобба-Дугласа}, которая описывает, как производственный выход зависит от входных факторов.

\textit{Определение} \textbf{Функция Кобба-Дугласа}  описывает связь между факторами производства и производственным результатом. Она используется для анализа того, как входные факторы влияют на объем производства.

\[ Q = A \cdot L^\alpha \cdot K^\beta, \]

где:\begin{itemize}
    \item \( Q \) - объем производства (выход, производственный результат);
    \item  \( L \) - количество труда (входной фактор);
    \item \( K \) - количество капитала (входной фактор);
    \item \( A \) - технический коэффициент (иногда интерпретируемый как общая производительность);
    \item \( \alpha \) и \( \beta \) - коэффициенты эластичности выхода по труду и капиталу соответственно.
\end{itemize}

Функция Кобба-Дугласа имеет следующие ключевые особенности:
\begin{itemize}
    \item Линейная в обоих аргументах: объем производства зависит от величины каждого входа в первой степени;
    \item Постоянные отдачи от масштаба: суммарная эластичность по входам (\( \alpha + \beta \)) равна единице, что означает, что при увеличении затрат на труд и капитал в одинаковой пропорции объем производства увеличивается в той же пропорции.
\end{itemize}
