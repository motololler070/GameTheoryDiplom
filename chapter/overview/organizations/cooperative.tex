\textit{Определение} \textbf{Кооперативная игра} - модель в теории игр, в которой участники  объединяют свои ресурсы и стремятся к совместной реализации общей цели. Каждый игрок может входить в различные коалиции с другими игроками, и выигрыш каждой коалиции зависит от участия и вклада ее членов.

Пусть дана кооперативная игра \( (N, v) \), где:
- \( N = \{1, 2, ..., n\} \) - множество игроков (участников игры).
- \( v \) - функция выигрыша (характеристическая функция), которая присваивает каждой коалиции \( S \subseteq N \) значение \( v(S) \), представляющее общую выгоду или выигрыш, получаемый этой коалицией.


Таким образом, для любой коалиции игроков \( S \subseteq N \), значение \( v(S) \) представляет собой объективную меру выгоды, которую эта коалиция может достичь совместными усилиями.


Математически, функция выигрыша \( v(S) \) должна удовлетворять следующим условиям:\
\begin{itemize}
    \item \( v(\emptyset) = 0 \) - выигрыш пустой коалиции (несуществующей) равен нулю.
\end{itemize} \( v(N) \) - выигрыш всего множества игроков равен общему выигрышу, получаемому от всей кооперативной игры.

\textit{Определение} \textbf{Вектор Шепли }- это метод распределения выигрыша в кооперативных играх, который учитывает вклад каждого игрока в формирование коалиций. 

Для расчета вектора Шепли \( (\phi_1, \phi_2, ..., \phi_n) \) для игры с \( n \) игроками, где \( \phi_i \) - это вклад игрока \( i \) в выигрыш коалиции, используется следующая формула:

\[
\phi_i = \sum_{S \subseteq N \setminus \{i\}} \frac{|S|! \cdot (n - |S| - 1)!}{n!} \cdot (v(S \cup \{i\}) - v(S))
\]

где:
- \( N \) - множество игроков в игре.
- \( v(S) \) - значение выигрыша для коалиции \( S \).
- \( |S| \) - количество игроков в коалиции \( S \).
- \( n \) - общее количество игроков в игре.

Эта формула описывает вклад каждого игрока \( i \) в выигрыш, учитывая все возможные коалиции, в которых он участвует. Вклад игрока \( i \) определяется разницей в выигрыше между коалициями, в которых игрок присутствует и отсутствует, взвешиваемой на фактор, учитывающий количество способов образования таких коалиций.
