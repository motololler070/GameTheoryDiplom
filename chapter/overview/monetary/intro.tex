В разделе рассмотрено определение блага, его формы в экономических и социальных постановках.

Основные показатели экономического благополучия:
\begin{itemize}
    \item Инфляция $\pi$: Измеряет изменение уровня цен за определённый период времени;
    \item Мультипликатор депозитов $m$: Отношение между депозитами и резервами банков;
    \item Ставка резервного обязательства $RR$: Процент средств, которые банки должны удерживать в резерве;
    \item Денежная база $MB$: Общая сумма денег в обращении плюс резервы банков;
    \item Денежное предложение $M$: Общая сумма денег, находящаяся в обращении в экономике;
    \item Процентная ставка $r$: Цена за заем капитала или доход от инвестиций;
    \item ВВП $Y$: Общая стоимость всех произведённых товаров и услуг за определённый период времени;
    \item Безработица $u$: Процент людей, которые активно ищут работу, но не могут её найти.
\end{itemize}



