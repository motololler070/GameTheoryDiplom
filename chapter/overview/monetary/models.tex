\textit{Определение }\textbf{Формула Фишера}

   \[
   i = r + \pi
   \]
   Эта формула устанавливает связь между номинальной процентной ставкой (\(i\)), реальной процентной ставкой (\(r\)) и ожидаемым уровнем инфляции (\(\pi\)). Она позволяет анализировать, как изменения в инфляции влияют на номинальные процентные ставки.


\textit{Определение }\textbf{Модель Филипса}
   \[
   \pi = \pi^e - \alpha(u - u^*),
   \]

   где:
\begin{itemize}
   \item \(\pi\) представляет уровень инфляции;
   \item \(\pi^e\) - ожидаемый уровень инфляции;
   \item  \(u^*\) - естественный уровень безработицы;
   \item \(u\) - уровень безработицы;
   \item \(\alpha\) - параметр, описывающий связь между инфляцией и безработицей.
\end{itemize} 



\textit{Определение } \textbf{Модель IS-LM}



   \[
   Y = C(Y - T) + I(r) + G
   \]
   \[
   M = L(Y, i),
   \]   
где:
\begin{itemize}
   \item \(Y\) - ВВП;
   \item  \(C\) - потребление;
   \item  \(T\) - налоги;
   \item \(I\) - инвестиции;
   \item  \(G\) - государственные расходы;
   \item  \(r\) - процентная ставка;
   \item  \(M\) - денежная масса;
   \item  \(L\) - спрос на деньги.
\end{itemize}

Модель иллюстрирует взаимосвязь между уровнем дохода, процентными ставками и денежным предложением в экономике.
