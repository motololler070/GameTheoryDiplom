% https://ru.wikipedia.org/wiki/%D0%A1%D1%87%D1%91%D1%82%D0%BD%D1%8B%D0%B5_%D0%B4%D0%B5%D0%BD%D1%8C%D0%B3%D0%B8

\textit{Определение} \textbf{Денежно-кредитная политика } 

Вопросы оптимального регулирования активно изучаются \cite{нэш2008идеальные}
Регулирование выполняется с помощью инструментов финансовой инженерии \cite{архипов2007предпосылки}. Среди них можно выделить управления ставкой рефинансирования, уровнем резервирования. 



Выделяют различные стратегии монетарной политики, различающиеся для стран с различным уровнем развития. 
\textit{Определение} \textbf{Инфляционное таргетирование} \cite{bernanke1997inflation} - монетарная стратегия, направленная на удержание целевого уровня инфляции. 

Исходя из экономической ситуации центральный банк использует инструменты. Такая стратегия выделяется прозрачностью намерений банка, политика регулирования однозначно интерпретируема намерением удержать ключевой показатель.

Существуют различные способы создания денежной массы.
\textit{Определение} \textbf{Монетизация долга} процесс при котором центральный банк приобретает государственные облигации на открытом рынке и финансирует. 

Таким образом, правительство приобретает возможности для покрытия бюджетных дефицитов без явной потребности в изменение налоговой политики или поиск займа.

