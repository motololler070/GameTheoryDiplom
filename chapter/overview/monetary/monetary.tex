\textit{Определение} \textbf{Денежно-кредитная политика } 




Вопросы оптимального реглуирования активно изучаются \cite{нэш2008идеальные}

Регулирование выполняется с помощью иснтрументов финансовой инженерии \cite{архипов2007предпосылки}. Среди них можно выделить управления ставкой рефинансирования, уровнем резервирования. 



Выделяют различные стратегии монетарной политики, различающиеся для стран с различным уровнем развития. 

\textit{Определение} \textbf{Инфляционное таргетирование} \cite{bernanke1997inflation} - монетарная стратегия, направленная на удержание целевого уровня инфляции. 

Исходя из экономической ситуации центральный банк использует инструменты. Такая стратегия выделяется прозрачностью намерений банка, политика регулирования однозначно интерпретируема намерением удержать ключевой показатель.



Существуют различныя способы создания денежной массы.

\textit{Определение} \textbf{Монетизация долга} процесс при котором центральный бнак приобретает государственные облигации на открытом рынке и финансирует. 

Таким образом, правительство приобетает возможности для покрытия бюджетных дефицитов без явной потребности в изменение налоговой политики или поиск займа.



