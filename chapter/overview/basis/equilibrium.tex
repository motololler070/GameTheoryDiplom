Равновесие по Нэшу - это концепция в теории игр, предложенная Джоном Нэшем, в которой каждый игрок в игре выбирает стратегию, которая является оптимальной для него, учитывая стратегии, выбранные остальными игроками. \cite{nash1950non}

Введем обозначение $s_{-i}$ используется для обозначения стратегий всех игроков, кроме игрока $i$. 

Пусть \(S_i\) - множество стратегий для игрока \(i\). Тогда профиль стратегий $(s_1^{*}, \dots,s_n^{*})$ является равновесным по Нэшу, если для каждого игрока $i$ выполняется:
$$
    u_i(s_i^{*},s_{-i}^{*}) \ge u_i(s_i,s_{-i}^{*}), \forall s_i \in S_i
$$

В равновесии по Нэшу каждый игрок выбирает свою стратегию, оптимизируя свой выигрыш, учитывая стратегии всех остальных игроков, и ни один игрок не имеет желания изменить свою стратегию, учитывая стратегии других игроков.
