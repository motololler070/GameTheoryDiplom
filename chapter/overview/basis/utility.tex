В теории игр утилитарность обычно описывается как числовая оценка, представляющая собой меру выигрыша или удовлетворенности, которую игрок получает от определенного исхода игры. Утилитарность используется для оценки предпочтений игроков и выражения их целей и интересов в контексте игры.

Формально, утилитарность игрока \(i\) обычно обозначается как \(u_i\) и является функцией от исхода игры, которая отображает каждый исход в числовое значение, представляющее собой удовлетворенность или ценность для игрока \(i\). Таким образом, если \(s\) - исход игры, то \(u_i(s)\) представляет утилитарное значение для игрока \(i\) от данного исхода.

Утилитарность игрока может зависеть от различных факторов, таких как его предпочтения, цели, ресурсы и контекст игры. Цель каждого игрока заключается в выборе стратегии, которая максимизирует его утилитарное значение, т.е. приносит ему наибольшую выгоду или удовлетворение в рамках заданной игры.
