
\textit{Определение} \textbf{Марковской цепью} называют случайный процесс представляющий последовательность, где вероятность следующего события зависит только текущего состояния системы. 

Таким образом, для системы с множеством состояний $S$ свойство марковости запишется следующим образом для $\forall S$:

$$
    P(x_{t+1}=i_{t+1}|X_t=i_t,\dots,x_0=i_T),
$$
 где $i_0,\dots,i_j \in S$  

\textit{Определение} \textbf{Стратегия} - это полный набор решений, которые игрок применяет при всевозможных ситуациях. Стратегия определяет набор действий игрока в любой момент времени и для каждого возможного сценария происходящего, способного привести к каждой ситуации.

\textit{Определение} \textbf{Чистой} называют стратегию, которая предполагает, что игрок принимает решение $a$ с вероятностью равной 1 $p(a)=1$. 

Иными словами, игрок использует только одно действие или одну последовательность действий в каждой возможной ситуации в игре.

\textit{Определение} \textbf{Доминантной} называют стратегию для игрока, которая гарантирует наибольший выигрыш вне зависимости от действий других игроков.



