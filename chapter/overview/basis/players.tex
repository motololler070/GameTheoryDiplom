Для изучения отношений и конфликтов интересов между двумя сторонами, одна из которых (принципал) нанимает другую (агента) для выполнения определенной задачи или достижения определенной цели. Принципал и агент - это две ключевые роли в теории игр.


\textit{Определение} \textbf{Принципал} -  лицо, которое делегирует выполнение задачи или принятие решений агенту. Принципал имеет определенные цели или интересы, которые он хочет достичь, и он нанимает агента для выполнения работы, связанной с этими целями. Принципал может контролировать или влиять на действия агента через различные механизмы стимулирования или контроля.

\textit{Определение}  \textbf{Агент} - сторона, которая нанимается принципалом для выполнения определенной работы или достижения цели. Агент действует от имени принципала, но может иметь свои собственные интересы или цели, которые могут не совпадать полностью с интересами принципала. В контексте теории принципал-агент, ключевыми аспектами являются мотивация агента и меры контроля принципала для обеспечения выполнения задачи в соответствии с его интересами.


\textit{Определение} \textbf{Рациональным} называют игрока описывается способный принимать решения, направленные на максимизацию его выигрыша или ожидаемой полезности, учитывая его предпочтения и информацию о среде.

Рациональность игрока является ключевым предположением в анализе теории игр и обычно используется в моделировании поведения игроков для прогнозирования и анализа исходов игр.


