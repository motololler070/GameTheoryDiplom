


В теории игр общественный выбор обычно определяется как процесс агрегирования предпочтений или мнений всех игроков с целью принятия коллективного решения. Этот процесс может быть использован для определения того, какая стратегия или исход будет выбрана в игре, учитывая предпочтения и интересы всех участников.

Для определения общественного выбора в теории игр часто используется понятие равновесия. Общественный выбор в этом контексте может быть понят, как такой исход игры, который учитывает стратегии всех игроков и удовлетворяет определенным критериям эффективности или справедливости.

Один из способов определения общественного выбора в теории игр - это использование концепции равновесия по Нэшу. Общественный выбор в этом случае может быть определен, как профиль стратегий, при котором ни один игрок не имеет стимула отклониться от своей стратегии, учитывая стратегии других игроков. Таким образом, общественный выбор в равновесии по Нэшу представляет собой некоторый устойчивый исход игры, который удовлетворяет критериям стабильности и эффективности.

Другие подходы к определению общественного выбора включают использование различных критериев, таких как социальная полезность, равенство или справедливость, в зависимости от контекста игры и поставленных целей.


Пусть дана игра с \(N\) игроками, где каждый игрок \(i\) выбирает свою стратегию из множества стратегий \(S_i\). Пусть также \(u_i(s_1, s_2, ..., s_N)\) - функция полезности игрока \(i\) относительно данного профиля стратегий \((s_1, s_2, ..., s_N)\).

Общественный выбор (общественное решение) - это такой профиль стратегий \((s_1^*, s_2^*, ..., s_N^*)\), где каждая стратегия \(s_i^*\) выбирается с учетом стратегий всех остальных игроков, таким образом, что:

\[ s_i^* \in \arg\max_{s_i \in S_i} u_i(s_i, s_{-i}^*), \]

для всех \(i = 1, 2, ..., N\), где \(s_{-i}^*\) - это профиль стратегий всех игроков, кроме \(i\).

Это означает, что каждый игрок выбирает свою стратегию, максимизируя свою полезность относительно стратегий остальных игроков. Общественное решение является таким профилем стратегий, при котором ни один игрок не имеет стимула отклониться от своей стратегии, учитывая выбор остальных игроков.
