Неполная информация о рыночной ситуации влияет на стратегии игроков.
Введем основные определения направления. 

\textit{Определение} \textbf{Cигнал} $s_i$ представляет собой информацию, которую игрок получает о состоянии игры или о стратегии другого игрока перед принятием своего решения. 

Сигналы могут оказывать влияние на стратегии игроков, изменяя их ожидания или оценки о ситуации в игре. Использование такой информации может быть ключевым для принятия оптимального решения в игре с неопределенностью или неполной информацией.




\textit{Определение} \textbf{Байесова игра}\cite{harsanyi1967games} \cite{harsanyi1968games}\cite{harsanyi1968bayesian} - это игра в теории игр, где игроки имеют неполную информацию о стратегиях и/или типах других игроков, но обладают апостериорными вероятностями о возможных значениях этой информации на основе предшествующей информации или наблюдений. Этот тип игр моделирует ситуации, в которых игроки принимают решения в условиях неопределенности и используют свои апостериорные вероятности для принятия решений.
В байесовской игре задается следующее:

\begin{itemize}
    \item Множество игроков: \( N = \{1, 2, ..., n\} \);
    \item Пространство типов игрока \(i\): \(T_i\);
    \item Пространство типов игрока \(i\): \(T_i\).
    \item Функция полезности игрока \(i\): \(u_i: S_1 \times T_1 \times ... \times S_n \times T_n \rightarrow \mathbb{R}\);
    \item Функции апостериорных вероятностей игроков: \(\pi_i: H_i \times S_{-i} \rightarrow [0, 1]\), где \(H_i\) - множество историй, наблюдаемых игроком \(i\);
    \item Пространство историй: \(H = \prod_{i=1}^{n} (S_i \times T_i)\).
\end{itemize}


Байесовская игра заключается в том, чтобы для каждого игрока \(i\) выбрать стратегию \(s_i \in S_i\) в зависимости от его типа \(t_i \in T_i\), учитывая его апостериорные вероятности \(\pi_i\). Ожидаемая полезность игрока \(i\) задается как:
\[ U_i(s_i, t_i) = \int_{H_i} u_i(s_i, t_i, s_{-i}, t_{-i}) \cdot \pi_i(h_i \mid s_{-i}, t_{-i}) \, dh_i ,\]


где \(s_{-i}\) и \(t_{-i}\) обозначают стратегии и типы остальных игроков, а \(h_i\) обозначает историю, наблюдаемую игроком \(i\).


\textit{Определение} \textbf{Равновесие по Байесу} - это набор стратегий, где ни один игрок не имеет стимула изменить свою стратегию, учитывая свои апостериорные вероятности и действия других игроков.
