\textit{Определение} \textbf{Гипотеза эффективности рынка} -  концепция в финансовой экономике, которая предполагает, что цены финансовых инструментов  полностью отражают всю доступную информацию о них. 

Подробный предметный обзор представлен в работе \cite{архипов2007предпосылки}. Частным следствие гипотезы является заключение, что на финансовых рынках невозможно получить выгоду от торговли на основе публично доступной информации, так как цены уже отражают всю доступную информацию:

\[ P_t = P_t' + \varepsilon_t \]
, где:
- \( P_t \) - текущая цена актива.
- \( P_t' \) - фундаментальная стоимость актива.
- \( \varepsilon_t \) - случайная ошибка или шум.


\textit{Определение} \textbf{Парадокс Гроссмана и Стиглица} \cite{grossman1980impossibility}



В реальных постановках крупные компании затрачивают значительные ресурсы на получение информаци о состоянии предметной области.


$$
    P_t = \mathrm{E}[V_t|I_{t-1}]
$$,

здесь $V_t$ представляет истинную стоимость.




Парадокс возникает вследствие того, что в $$


