\textit{Определение} \textbf{Гипотеза эффективности рынка} -  концепция в финансовой экономике,
 которая предполагает, что цены финансовых инструментов полностью отражают всю доступную информацию о них. 

Подробный предметный обзор представлен в работе \cite{архипов2007предпосылки}. 
Частным следствие гипотезы является заключение, что на финансовых рынках невозможно получить выгоду от торговли на основе публично доступной информации, 
так как цены уже отражают всю доступную информацию:
\begin{equation}
     P_t = P_t' + \varepsilon_t,
\end{equation}
где \begin{itemize}
    \item \( P_t \) - текущая цена актива.
    \item \( P_t' \) - фундаментальная стоимость актива.
    \item \( \varepsilon_t \) - случайная ошибка или шум.
\end{itemize}



\textit{Определение} \textbf{Парадокс Гроссмана и Стиглица} \cite{grossman1980impossibility}



В реальных постановках крупные компании затрачивают значительные ресурсы на получение информации о состоянии предметной области.

\begin{equation}
    P_t = \mathrm{E}[V_t|I_{t-1}],
\end{equation}
здесь $V_t$ представляет истинную стоимость.


