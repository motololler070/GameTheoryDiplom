\textit{Моральный  риск} в сделках возникает вследствие невозможности участников в полной мере быть уверенными в отсутствие недобросовестных намерений у второй стороны.

Классическим примером в литературе является заключение страхового контракта. Застрахованный клиент имеет меньшие риски в несчастных случаях, в следствие может вести более , ведущие к потерям.

Проблема также часто встречается в постановках кредитования и денежного вознагражения агентов. Разрешением института репутации. 

В постановках кредитования и страхования репутация проверяется в автоматическом порядке с помощью интерпретируемых классификаторов, основанных на алгоритме машинного обучения.

Пусть $x$ -вектор переменных клиента, $D$ - обучающая выборка,  $\theta$ -  параметры обучения классификатора. Тогда при обучении классификатор как правило максимизируется правдоподобие
$$
    \max_\theta  p(D|\theta) 
$$


\textit{Опредение} \textbf{Издержки}\сite{}


\textit{Определение} \textbf{Вероятностные графические модели} \сite{koller2009probabilistic}



Неблагоприятный отбор  \cite{akerlof1978market}