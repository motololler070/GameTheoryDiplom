% https://www.youtube.com/watch?v=Xn09KsMNLrs

\textbf{Теорема}(Об огибающей) Пусть $f(x,\alpha)$  функции $\mathbb{R}^{n+l}$. Зададим оптимизационную постановку с помощью набора из $m$ непрерывно дифференцируемых функций $g_j(x,\alpha)$:

$$
    \max_{x} f(x,\alpha), g_j(x,\alpha) \ge 0, j=1,2,\dots,m, x\ge 0
$$
Оптимизационный лангранжиан для постановки запишется как:

$$
    \mathcal{L}^*(\alpha)=f(x^*(\alpha),\alpha) +\lambda^*(\alpha) \cdot g(x^*(\alpha),\alpha)
$$

Целевая функция задана как $V(\alpha)=f(x^*(\alpha),\alpha)$ , где $x^*$ - оптимум при заданных параметрах.

$$
    \frac{\partial V(\alpha)}{\partial \alpha_k} = \frac{\partial \mathcal{L}^*(\alpha)}{\partial \alpha_k}, k=1,2,\dots,l
$$

На практике теорема о огибающей используется для анализа оптимизационных постановок, в которых возможное малое изменение параметра $\alpha$ влияет на значение целевой функции $f(x)$.

\textbf{Теорема} Пусть $f(x,\cdot)$ абсолютна непрерывна при всех $x \in X$. Предположим, что есть интегрируема функция $b(t):[0,1] \rightarrow \mathbb{R}$ , ограничивающая по модулю $f(t)$ в каждый момент времени $|f_t(x,t)| \le b(t)$ для любых $x\in X$. Тогда $V$ абсолютна непрерывна. Если, 

$$
    V(t) = V(0) + \int_0^t f_t(x^*(s),s) ds
$$

Теорема позволяет изучать смещения оптимума при изменении инструментального параметра.


Игрок определяется типом {$t$}, являющегося элементом множества $T$. Значение определяется в контексте как тип участника аукциона, денежной политики или рыночной структуры.

В работе Майерсона 1981 г. \cite{myerson1983mechanism} тип означает степень информированности агента о ставках прочих игроков.




