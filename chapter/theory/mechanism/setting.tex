

% https://www.youtube.com/watch?v=j5VmiSfVvgI&list=PLeY-lFPWgBTgX_gjAMIBrFxxTB_Sy44i_&index=2&t=238s
\textit{Определение 1} \textbf{Байесова игра} это набор $(N)$ таких что:

\begin{itemize}
    \item $\mathbf{N}$ конечное множество агентов $n$;
    \item $O$ множество исходов;
    \item $\Theta = \Theta_1 \times \Theta_2 \dots \Theta_n $ множество ;
    \item $u = (u_1, \dots, u_N)$, где $u_i: O \times \Theta \rightarrow \mathrm{R}$  функция полезности для игрока $i$.
\end{itemize}


%Доверие

%Максимум выгоды

%Монотонность по возвратам


\textit{Определение 2} \textbf{Механизм} для байесовой игры это пара $(A,M)$, где:
\begin{itemize}
    \item $A = A_1 \times \dots A_n$ набор действий доступный агенту i;
    \item $M: A \rightarrow \Pi(O)$ соединяет действия с распределением возможностей.
\end{itemize}


Таким образом, архитектору необходимо определить набор возможных действий для агентов,  их связь с исходами и полезностью для агентов. Архитектору не доступно изменений самой среды и действий.


Задача заключается в подборе механизма, задающего для рациональных агентов надлежащее для задачи поведение - максимизацию общественного блага.

\textit{Определение 3} В заданной байесовой игре, механизм является \textbf{воплощением доминантной стратегии} социального выбора функции C, если для любого вектора полезности $u$, у игры есть равновесие в доминантной стратегии, и для любого равновесия $a^*$ выполняется $M(a^*) = C(u)$.

Потенциальной проблемой на практике является отсутствие единственного равновесия. Как правило, разрешение заключается в дополнении условий симметричным ограничением.\cite{milgrom1982theory}

Воплощения различаются по природе:
\begin{itemize}
\item прямое. Все агенты одновременно посылают единственное сообщение в координационный центр.
\item непрямое. Агенты посылают ряд сообщений.
\end{itemize}


