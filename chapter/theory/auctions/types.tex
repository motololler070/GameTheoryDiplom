Аукционы различаются по форме организации, открытости ставок игроков, правилам назначения цены и порядком задания ставки.
Опишем исторически сложившееся аукционы: 
\begin{itemize}
    \item \textbf{Японский}. Продавец начинает с высокой цены и постепенно снижает ее, а покупатели могут выйти из аукциона, когда цена становится слишком низкой для них. Покупатель, который первым подтверждает цену, покупает товар по этой цене.
    \item \textbf{Английский}. Продавец начинает с низкой цены, а затем плавно повышает ее. Покупатели делают ставки, предлагая более высокие цены за товар. Аукцион заканчивается, когда нет новых ставок. Покупатель, сделавший последнюю ставку, покупает товар по этой цене.
    \item \textbf{Голладнский}. Похож на японский, но процесс изменения цены происходит гораздо медленее, чем в голландском аукционе.
\end{itemize}

Аукционы используются в различных сферах, и каждый из них имеет свои преимущества и недостатки. 

По степени открытости:

\textit{Определение} В \textbf{открытом} аукционе все участники могут видеть ставки других участников.

Пусть \( b_i \) обозначает ставку, сделанную участником \( i \), \( S \) - множество всех участников. Тогда открытый аукцион можно представить как функцию, которая сопоставляет множеству ставок \( \{b_i\}_{i \in S} \) победителя аукциона.

\textit{Определение} В \textbf{закрытом} аукционе ставки участников остаются скрытыми от других участников.
 
 Пусть \( b_i \) обозначает ставку, сделанную участником \( i \), \( S \) - множество всех участников. Тогда закрытый аукцион можно представить как функцию, которая сопоставляет каждому участнику \( i \) его собственную ставку \( b_i \) и победителя аукциона.
