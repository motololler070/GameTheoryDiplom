% https://www.youtube.com/watch?v=FcV2RwvGci4&list=PLeY-lFPWgBTjOH5_6YXeYQIbL6N_WtX9G&index=6
% https://en.wikipedia.org/wiki/Bayesian_efficiency1

\textbf{Теорема 1} \textit{Характеристика Байес-Нэшевого равновесия}

Стратегическое поведение $s$ для выборки из $n$ агентов обладает следующими свойствами:

\begin{itemize}
    \item монотонность $\xi_i(v_i|s)$ монотонно возрастающая;
    \item неизменность платежа $p_i(v_i|s) = v_i \xi_i (v_i|s)  \int_0^{v_i} \xi_i(z|s) dz + p_i(0|s)$.
\end{itemize}


\textbf{Теорема 2} \textit{Эквивалетность выручки}

Предположим, что каждый из $n$ риск-нейтральных агентов имеет личное предпочтение относительно предмета аукциона из функции распределения $F$. Тогда для любых двух механизмов аукционов:

\begin{itemize}
    \item в равновесном состоянии лот распределен одинаково;
    \item любой агент с оценкой товара 0 имеет полезность 0.
\end{itemize}
Таким образом, ожидаемая выручка и ставка участника с оценкой $V$ будет иметь одинаковую выплату

\textit{Доказательство}

$k$-ый момент распределения равен матожиданию $k$-ого наибольшего  элемента из  выборки мощностью $n$. 

Для $n$ независимых из распределения $k$-ый момент равен

$$
    \frac{n+1-k}{n+1} v_{max}
$$

Следовательно для аукциона второй цены:

$$
    \frac{n-1}{n+1} v_{max}
$$

Каждая симметричная игра имеет симметричное равновесие.  Для симметричного равновесия верно, что наибольшая ставка равна наибольшей оценки полезности лота.