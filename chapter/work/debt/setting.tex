Кредит берется на покрытие издержек производства и фонд оплаты труда. Считаем, что поставщики закупают сырье по равной рыночной цене $FC$.  Ставки юридического кредита также примем равными $i$ и приблизительно соответствующей ставке рефинансирования $i_{\text{ref}}$. Считаем, что поставщик до сбыта товара платит по кредитным обязательствам только минимальный дифференцированный платеж, соответствующий $i$ $FC_i$.

В случае неудачи поставщик повторно участвует в аукционе реализуя произведенную продукцию
Лот аукциона согласно ФЗ 44 и 223 неделим и обязан быть реализован одним поставщиком. Будем считать, что закупка выполняется без экспертной оценки, победитель аукциона определяется лучшей ценой.  Также введем параметр ликвидности рынка , задающий число шагов аукциона на поиск нового лота.

Торги выполняются последовательно между двумя поставщиками. Сначала школа назначает цену лота $K_0$, после поставщики последовательно предлагают скидку  $M_i$.  Право первого хода определяется по жребию, число этапов аукциона без ограничения общности можно считать неограниченным. Агенты конкурируют максимизируя  прибыль поставки i:
$$
    max i= K0-M -FC -FC i (+ ), 
$$
,где $M$ - итоговая скидка,  - число шагов проведенных в аукционах,  - число участий в  аукционах  до реализации продукции.