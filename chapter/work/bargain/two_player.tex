Среди бюджетных ассигнований, согласно , помимо основных также есть и стимулирующие выплаты для руководителей образовательных учреждений. Их разделение выполняется органом местного самоуправления образовательных организаций на основе форм независимой оценки качества образования.
Решение о размере премирования выносится в соответствии с разработанным нормативным актом органа местного самоуправления. 

В рамках раздела мы приведем описание подход из теории игр для моделирование раздела ограниченной суммы между двумя школами.
Раздел будет проходить в ходе последовательных торгов, определяющих пропорцию раздела бюджет, который выделит надзирающий орган.
Таким образом мы определим места пристального внимания при составлении нормативных актов, определим качества игроков, позволяющие им приобрести большую долю при споре.

Для исследования раздела между двумя игроками используется модель последовательных торгов Ариэля Рубенштейна.
Согласно модели участники последовательно предлагают долю раздела, заявленного бюджета.
Если оппонент не согласен с предложением, он его отклоняет и выдвигает свой способ раздела. 
Игроки, заинтересованы в скорейшем разрешении спора, и потому обладают коэффициентом дисконта  $\in (0,1)$, определяющий насколько ценно для игрока время, затраченное на переговоры.
Таким образом, функция полезности игрока спустя ход $U_{T+1}$ составляет $U_T$ при равных условиях раздела.
Изначально рассмотрим ситуацию торгов между двумя игроками. Функция полезности с учетом дисконтирования на временном шаге $T$ в отсутствии раздела для игроков задается как:
\begin{equation}
	\begin{aligned}
		&U_1^{(T)}= \frac{1}{T-1r_T}  \\
		&U_2^{(T)}=2\frac{T-1}{(1-rT)},
	\end{aligned}
\end{equation}
,$r_T$- доля раздела первого игрока, предложенная на шаге $T$.
Оптимальная стратегия для игрока предлагать оппоненту минимальный раздел, на который он согласится. 
\begin{equation}
	\begin{aligned}
		&1 - r_T  \le 2(1-r_T+1)  \\
		&1 r_T  \le r_T+1 
	\end{aligned}
\end{equation}
Перепишем неравенства и получим:
\begin{equation}
	\begin{aligned}
		&1-2(1-r_T+1) \le r_T \le r_{T}+1 /1 \\
		&1-1-r_T2r_T+1r_T1 
	\end{aligned}
\end{equation}
Получаем неравенства на $r_T$, $r_{T+1}$ :
\begin{equation}
	\begin{aligned}
		&r_T \le 1-2 1-12 ;1 -r_T 2(1-1)1-1 2 \\
		&r_{T+1} \ge  1(1-2)1-12 ; 1 -r_T+1 1-11-1 2,
	\end{aligned}
\end{equation}
В заданных условиях игрок 1 предлагает раздел $r1=1-2 1-12$  и соглашается не менее чем на $r2 1(1-2)1-12$ .
Заметим, что сторона, выдвигающая предложение, имеет преимущество в величину дисконта. Также можем сделать логичный вывод, что с ростом собственного коэффициента дисконта величина раздела увеличивается. При росте дисконта противника - уменьшается. Таким образом, результат спора определяется инициативностью и терпеливостью участников.
Также полученная модель может быть использована для описания взаимодействие в сфере образования. Так управляющие лица, которыми могут быть представлены муниципальные органы, так и министерства, используют преимущество первого хода, задавая правила распределения бюджета, определяющие размер ассигнования. Также с целью снижения коэффициента дисконта подрядчика - его возможности для обсуждения размеров бюджетирования - задаются сжатые сроки на подачи заявлений, формирование отчетов, необходимых для получения финансирования. Таким образом, управляющий орган получает возможность распределять ограниченные ресурсы согласно делегированной ему стратегии бюджетирования.  


