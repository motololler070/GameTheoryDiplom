Модель торгов между n игроками
Для случая торгов между N участниками используется модель Бэрона и Фереджона \cite{baron1989bargaining}. Право выдвигать предложение определяются случайно с равной вероятностью для каждого игрока 1N. Решение о разделе принимается, если K игроков его поддерживают, иначе право на голосование вновь случайно распределяется между игроками.  
Полное решение в предположениях марковости и симметричности решения описано в [2]. Автор предлагает ввести две контрольные величины для игрока: R — ожидаемый выигрыш игрока, предлагающего раздел; r — ожидаемый выигрыш игрока, не предлагающего раздел. Тогда с вероятностью $\frac{1}{N}$ игрок будет предлагать дележ и получит выигрыш $R$, иначе с вероятностью $\frac{N-1}{N}$ его выигрыш составит $r$. Так что для того, чтобы игрок согласился на предлагаемый дележ в данный момент времени, необходимо, чтобы ему предложили как минимум $\delta( \frac{1}{N} R  + r\frac{N-1}{N})$ .
Тогда сформированное предложение будет иметь вид:
\begin{itemize}
    \item $1 - (K - 1)\delta( R \frac{1}{N} + r\frac{N-1}{N} )$ - предлагающему
    \item $\delta( R 1N + r\frac{N-1}{N})$ - $K-1$ игрокам, которые должны одобрить раздел 
    \item $0$ - каждому из оставшихся $N-K$ игроков.
\end{itemize}


При этом все $N-1$ игроков должны иметь одинаковую вероятность быть включенными в число $K-1$ игроков, которые одобрят дележ: $\frac{K-1}{N-1}$. Приравнивая ожидаемый выигрыш к предложениям получаем систему:

\begin{equation}
    \begin{cases}
        R = 1 - (K - 1) \delta ( R 1N + r\frac{N-1}{N} ) \\
        r= \delta \frac{K-1}{N-1}( R 1N + r\frac{N-1}{N} )
    \end{cases}   
\end{equation}


Решая систему, получаем $R$ и $r$:
\begin{equation}
    \begin{cases}
        R = N-\delta \frac{K-1}{N} \\
        r= \delta \frac{K-1}{N}(N-1) 
    \end{cases}   
\end{equation}

Тогда выигрыши игроков составят:
\begin{itemize}
    \item $N-\delta\frac{K-1}{N}$ - предлагающему
    \item N - K-1 игрокам, которые получили право одобрить раздел
    \item $0$ -  каждому из оставшихся $N-K$ игроков 
\end{itemize}


Приведем выводы, которые следуют из заявленной модели. Преимуществом при разделении ресурсов обладает сторона, выдвигающая решение. Для привлечения сторонников она выдвигает предложения, выгода которых обладает стохастической природой. Таким образом, возможна агитация каждого участника без явного формирования выгодополучателей - сговора. 

Заметим, что выгода предлагающей стороны убывает с числом K и коэффициентом дисконтирования . При этом равный раздел между сторонами возможен лишь при K=N и =1. 

Модель Бэрона и Фереджона удобна для описания органов самоуправления. Опишем её  применение для объединения школ муниципалитета, имеющих в среднем равные успехи в образование. Определение руководителя самоуправления согласно постановке задачи имеет стохастическую природу - результаты образовательных учреждений в среднем равны. Таким образом, каждый представитель школы имеет равные шансы на распределение премиального бюджета. Одним из целевых инструментов распределения в сфере образования является олимпиады. Премии выделяются школам с наибольшим числом победителей. В условиях равной успеваемости воспитанников школ статус победителя также равновероятен для всех участников. Для организации события руководитель использует делегированный бюджет, приобретая оборудование, дополнительно стимулируя сотрудников своего учебного заведения на составление заданий и присутствие на мероприятии, рекламируя олимпиаду. Таким образом, создаются все необходимые компоненты модели Бэрона и Фереджона:
равные возможности на предложение раздела
стохастическая природа распределения бюджета между игроками для легитимизации своего предложения
преимущество руководителя раздела