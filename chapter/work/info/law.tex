Объем рынка государственных учреждений, реализующих общеобразовательные программы, составляет миллиарды рублей. Реформы бюджетирования образовательных организаций позволяют совершенствовать профессиональную компетенцию молодых кадров, создавать новые педагогические практики, повышать удовлетворенность работой педагогов и руководителей образовательного процесса.  
Закупки в сфере образования могут осуществляться через конкурс согласно федеральным законам №239 и №44 и региональным нормативным ограничениям. Закупка осуществляется либо напрямую, либо через специальные механизмы государственных закупок. Чаще всего, в закупку вступают несколько ключевых поставщиков, которые конкурируют за поставку, предлагая наименьшую цену. 
Основным источником финансирования образовательных учреждений в настоящее время являются бюджетные ассигнования, рассчитанные на основе стоимости обучения обучающегося [3][4]. Согласно ФГОС в стоимость обучения включается:
\begin{itemize}
    \item форма обучения;
    \item тип образовательной организации;
    \item сетевая формы реализации образовательных программ, образовательных технологий;
    \item специальные условия получения образования обучающимися с ОВЗ;
    \item  обеспечение дополнительного профессионального образования педагогическим работникам;
    \item обеспечение безопасных условий обучения и воспитания;
охрана здоровья обучающихся;
    \item другие особенности.
\end{itemize}

Распределение бюджета согласно [5] выполняется на трех уровнях:

\begin{itemize}
    \item федеральный;
    \item региональный; 
    \item муниципальный.
\end{itemize}

Объем субвенции, передаваемой местному бюджету из бюджета субъекта РФ на реализацию государственного стандарта задаётся формулой:

$$
    \text{Ф}_\text{гс} = \text{N}_\text{c} \times \text{У}_\text{с} + \text{N}_\text{г} \times \text{У}_\text{г},
$$
где:
\begin{itemize}
    \item Nc - региональный расчетный подушевой норматив для сельской местности;
    \item Nг- региональный расчетный подушевой норматив для городской местности;
    \item Ус - количество сельских учащихся в данном муниципальном образовании;
    \item Уг - количество городских учащихся в данном муниципальном образовании.
\end{itemize}
             
                 
Объем средств, выделяемых образовательному учреждению рассчитывается по следующей формуле:

$$
    \text{Ф}= \text{N} \times \text{П} \times \text{У},
$$    
где:
\begin{itemize}
    \item  N - региональный расчетный подушевой норматив;
    \item У - число обучающихся в образовательном учреждении;
    \item П - поправочный коэффициент, установленный для данного образовательного учреждения.
\end{itemize}
            
Образовательное учреждение, самостоятельно определяет распределение на материально-техническое обеспечение и заработную плату работников образовательного учреждения, в том числе надбавки и доплаты к должностным окладам.
Таким образом, школа конкурирует за бюджет определенного уровня в зависимости от уровня подчинения школы.
Объем финансирования конкретных образовательных учреждений может быть пересмотрен путем корректировки поправочного коэффициента.
Выделяют также дополнительные источники финансирования:
\begin{itemize}
    \item платные дополнительные образовательные услуги;
    \item предпринимательская деятельность; 
    \item налоговые льготы;
    \item средства спонсоров;
    \item добровольные пожертвования родителей.
\end{itemize}
 
Среди них можно выделить родительскую плату за услуги учреждений в сфере обучения и оздоровления детей, благотворительность и выручку школьных столовых.
Заметим, что независимо от количества привлеченных средств бюджетное финансирование образовательного учреждения не снижается.
