В работе разобраны случаи аукциона при частичной и полной информированности игроков о значениях функциях полезности прочих игроков. Показано, что для случая полной информированности  в условиях достаточно конкурентного рынка скидка, предоставляемая заказчику, не отличается от закрытого аукциона. При привлечении к аукциону достаточного количества равных по  поставщиков, игроки будут выдвигать предложения по себестоимости независимо от их осведомленности. Ситуация частичной информированности может различаться в зависимости от значения функции полезности поставщика. Для приобретения выгоды информированному поставщику необходимо быть конкурентным и участвовать в сделках с малым числом игроком. В противном случае дополнительная информация не даёт преимуществ. Также разобран случай распределения бюджета в ходе последовательных торгов между участниками. Приведены модели, описывающие раздел между двумя и n-игроками. Разобраны способы применения моделирования для практик организации органов самоуправления и выделения финансирования. Показано, что в заданных условиях приобретает преимущество сторона, первая выдвигающая предположение и с большим терпением подходящая к разрешению спора. 